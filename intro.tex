\chapter{Introduction}\label{ch:intro}

Molten salt reactors ( MSRs ) first gained attention in the 1950s as a possible
high power density propulsion source for a nuclear powered aircraft. While this
idea never took flight the possibility for liquid molten salts as fuel for a
nuclear reactor did in the form of the Molten Salt Reactor Experiment ( MSRE )
conducted at Oak Ridge National Laboratory ( ORNL ) during the 1960s and 70s 
\cite{ORNL-MSRE}. Following an institutional redirection away from the MSRE
program molten salt studies continued at ORNL under the Molten Salt Breeder
Reactor program through the 1970s and into the 80s \cite{ORNL-MSBR}. After a
brief lull in global interest during the late 80s and early 90s studies in MSRs
began again in ernest in Europe begining with the SAMOFAR program and continuing
today under the ALISIA program \cite{SAMOFAR}. Since this time global interest
in MSRs has continued to grow with concepts coming from all corners of the
globe from both industry and governments: the Molten Salt Actinide Recycler and
Transmuter ( MOSART ) out of Russia, the FUJI series of reactors from Japan, 
the Liquid Fueled Thorium Molten Salt Reactor ( LF-TMSR ) out of China, the
Molten Salt Fast Chloride Reactor ( MCFR ) from TerraPower in the United States,
and many others.

This global interest is due to molten salt's many promising chracteristics such
as high operating temperatures, low operating pressures ( near atmospheric in
most cases ), excellent safety charachteristics, and high fuel utilization just
to name the top few. Many of these same characteristics create obstacles to 
modelling MSRs with today's computational tools due to the many physical
differences between MSRs and todays more numerous solid fuel light water
reactors. The key differnce and originator of the challenges is that the fuel
in a MSR is liquid and often flows through the core of the vessel and through
a heat exchanger. This flow of fuel induces many effects such as the drift of
delayed neutron precurssors and the bubbling out of gaseous fission products,
most notably the xenon isotopes.

Ivestigations of any nuclear reactor will include an anlysis of the proposed
fuel cycle. This is accomplished through coupling a transport code with a
nuclear depletion code. Investigations of MSR fuel cycles are more challenging
than those of their solid fuel counterparts due to a number of phenomenon: the
removal of various elements through natural processes such as bubbling and
platting, and the incorporation of operator actions on the reactor fuel stream.
Unlike light water reactors MSRs tend to operate near atmospheric pressure and
their flowing fuel allows for the continual addition or removal of chemical
species during reactor operation. This feature is exploited in two key ways.

First this feature allows MSRs to operate with a
low excess reactivity as fissionable material may easily be added during
operation. Secondly this feature allows the reactor operator to make adjustments
to the liquid fuel salt composition. This is of critical importance as the 
liquid fuel salt in a MSR will have some desired chemical state
in which the operator would like to maintain the salt. This is important for two
key reasons; first to prevent salt components from percipitating out of solution
should they exceed their solubility limits, and second the corrosion rate of the
specialty structural nickel/iron alloys is on the order of
micro-meters per year when the reduction potential of the salt is kept near
a specific value while a slight deviation from this value can raise corrosion
rates to centimeters per year \cite{Corrosion}. 
As such, it is in the operator's best interest to keep a tight control over 
the MSR fuel salt.

Capturing all of these phenomenon in a single nuclear fuel depletion code is
non-trivial and raises the question, how does the fuel salt composition in a 
molten salt fueled reactor change over time in response to both nuclear fuel
burnup and the reactor operator actions? Addressing this question is the goal
of the work presented herein. 

\subsection{Framing the problem}

The goal is to create a nuclear fuel depletion model which accounts for the
physical phenomenon acting on the liquid fuel of a MSR with a flowing fuel core.
Perhaps the least quanitifed unknown is that of the operator's actions. What are
the operators objectives? What tools does the operator have at their disposal?
What are the measures by which the operator assess their actions? As in most
simulation problems the variable of most varience here is the human. To
approximate the human in this model they are repalaced with a linear opimization
routine. If the constrainsts facing the operator may be approximated with linear
 relationships and if the operator's desires may be approximated as minimizing
or maximizing a given value then the choices of the operator may be predicted
via linear optimization. Ensuring that the problem constraints and optimization 
targets are all linear strikes a compromise between the ease and reproducibility
of the optimization solution and the model's adherence to the physical 
phenomenon being simulated.

One goal an operator is stipulated to have is keeping specific chemical
species within the fuel salt at some relative proportion to some other
specific chemical species. This goal arrises from the variable solubility of
chemical species within a given salt mixture - a solubility which changes
based both on temperature and the relative proportion of other chemical species.
A related objective an operator is stipulated to have is to maintain the fuel
salt reuction-oxidation (redox) potential at some desired value for corrosion
prevention as mentioned above. An important goal for all nuclear reactor
operators is to maintain the multiplication value of the reacting system. These
considerations here form the general operational contraints of a MSR. A
method for simulating the fuel cycle of a MSR must account for all these
considerations as well as for the physical phenomenon which act on MSRs
uniquely in reference to their solid-fuel counterparts.

Various approaches have been made to create a methodology for simulating MSR
fuel cyclels. Aufiero approximates nuclear criticality as only being dependent 
on two isotopes, chosen by the user which can be fed and removed according to 
the deviation from criticality. To handle salt species considerations Aufiero
ignores them, simply removing lithium for each fission product produced and
adding it for each fission product removed \cite{Aufiero}. Betlzer takes a more
nuanced approach using mass limits on chemical species to spur the injection or
removal of material
in a MSR; however, only one isotope is allowed to be specified for nuclear
criticality control \cite{Betzler}. Ridley ignores salt contituent balance
but rather choses to deal with the maintence of the redox potential only by
allowing the injection or removal of fluorine as tailored to a MSR using
\ce{LiF-BeF_{2}} eutectic salt. With regards to criticality control Ridley
uses a weighted difference scheme similar to Aufiero but only allowing the use
of Uranium 235 and 238 \cite{Ridley}. 

That which distinguishes this work from its peers is the way in which human 
actions on the reactor are simulated, via linear optimization in consideration
of all contraints at once, and by the completness of both the physical model
and this models integration into a simulation suite.

In the following chapters ADER, the Advanced Depletion Extension for
Reprocessing, is introduced. First the theory on which ADER is based is
presented in chapter \ref{ch:meth} along with its integration into the reactor
physics monte-carlo code SERPENT 2 \cite{Jaakko}. In chapter
\ref{ch:simul} the capabilities of ADER are investigated in relation to a
hypothetical MSR fuel cycle. In chapter \ref{ch:con} the effects of ADER on
MSR fuel cycle modelling are presented as well as recommendations for next
steps. 
