\chapter{Conclusions and Future Work}
\label{ch:conc}

In the last twenty years molten salt reactors have seen an incredible surge of
global attention as evidenced in such international programs as MOSART, ALISIA,
and the LF-TMSR out of China, as well as increased research attention as
evidenced by the continuing success of Oak Ridge National Laboratory's MSR
workshop which now averages over 400 attendees from across the globe yearly. 
While significant investment and progress has been made in nearly all aspects
of MSR development from material selection, salt purification and property
measurement, licensing and general reactor design, comparatively little
attention has been paid to fuel cycle analysis. The majority of attempts in
this area have fallen short of producing widely applicable results largely due
to the limiting effects of specific assumptions such as the salt species
involved or the coarseness of the solution. In this work a method and
implementation of a general approach for modelling and evaluating molten salt
reactor fuel cycles is proposed. This approach is based upon a linear
optimization routine designed to approximate the limitations of the chemistry
and nuclear concerns of reactor operations while optimizing the driving concerns
of the reactor operators. This approach, named ADER for the Advanced Depletion
Extension for Reprocessing, is built into the reactor physics monte-carlo code
SERPENT 2. ADER is found to have wide support for a broad range of possible
simulation parameters while the linearization scheme is found to produce
physically sensible results. While the theory and algorithm are found to be
sound, the specific implementation has been discovered to suffer from
significant and deleterious numerical instability within the 
linear optimization solver.

In its present state the usefulness of ADER is concerningly compromised and
the implementation as a whole is drastically limited in the range of simulations
it can execute with no method a priori to establish feasibility. Any future
work going forth on this project would need to begin with the implementation
of a floating-point quadruple-precision linear optimization solver in the place
of the current CLP implementation. Following such a development it is expected
that ADER will be capable of simulating the wide variety of parameters which its
input and structure allow. Additional improvements could be found through the
incorporation of an iteration scheme whereupon the effects of nuclear burnup
on the isotopics of the fuel over a burnup step can be approximated as a
proportional removal stream on the whole system in the linear optimization
scheme such that the approximated total effects of nuclear burnup are considered
in material optimization. Additional improvements may be found in abandoning the
linear optimization solver all together and rather moving towards a gradient
approach from which an interaction scheme to more closely approximate the 
effects of streams on nuclear criticality could be arrived at. 

Overall the impact of ADER is unclear and will only be known in the future.
Given ADER's extensive test suite, documentation, and modular construction, it
is not unimaginable that the above mentioned improvements may one day be made.
In such a future the impact of ADER would certainly be greater than it is today.
Despite this short coming ADER has shown that a linear optimization scheme can
be effectively applied to the chemistry, nuclear, and operational concerns of a
molten salt reactor in such a way as to predict the future fuel composition and 
nuclear characteristics of the system.
