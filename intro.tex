\chapter{Introduction}\label{ch:intro}

Molten salt reactors ( MSRs ) first gained attention in the 1950s as a possible
high power density propulsion source for a nuclear powered aircraft. While this
idea never took flight the possibility for liquid molten salts as fuel for a
nuclear reactor did in the form of the Molten Salt Reactor Experiment ( MSRE )
conducted at Oak Ridge National Laboratory ( ORNL ) during the 1960s and 70s 
\cite{ORNL-MSRE}. Following an institutional redirection away from the MSRE
program molten salt studies continued at ORNL under the Molten Salt Breeder
Reactor program through the 1970s and into the 80s \cite{ORNL-MSBR}. After a
brief lull in global interest during the late 80s and early 90s studies in MSRs
began again in earnest in Europe beginning with the SAMOFAR program and continuing
today under the ALISIA program \cite{SAMOFAR}. Since this time global interest
in MSRs has continued to grow with concepts coming from all corners of the
globe from both industry and governments: the Molten Salt Actinide Recycler and
Transmuter ( MOSART ) out of Russia, the FUJI series of reactors from Japan, 
the Liquid Fueled Thorium Molten Salt Reactor ( LF-TMSR ) out of China, the
Molten Salt Fast Chloride Reactor ( MCFR ) from TerraPower in the United States,
and many others.

This global interest is due to molten salt's many promising characteristics such
as high operating temperatures, low operating pressures ( near atmospheric in
most cases ), excellent safety characteristics, and high fuel utilization just
to name the top few. Many of these same characteristics create obstacles to 
modelling MSRs with today's computational tools due to the many physical
differences between MSRs and today's more numerous solid fuel light water
reactors. The key difference and originator of the challenges is that the fuel
in a MSR is liquid and often flows through the core of the vessel and through
a heat exchanger. This flow of fuel induces many effects such as the drift of
delayed neutron precursors and the bubbling out of gaseous fission products,
most notably the xenon isotopes.

Investigations of any nuclear reactor will include an analysis of the proposed
fuel cycle. This is accomplished through coupling a transport code with a
nuclear depletion code. Investigations of MSR fuel cycles are more challenging
than those of their solid fuel counterparts due to a number of phenomena: the
removal of various elements through natural processes such as bubbling and
platting, and the incorporation of operator actions on the reactor fuel stream.
Unlike light water reactors MSRs tend to operate near atmospheric pressure and
their flowing fuel allows for the continual addition or removal of chemical
species during reactor operation. This feature is exploited in two key ways.

First this feature allows MSRs to operate with a
low excess reactivity as fissionable material may easily be added during
operation. Secondly this feature allows the reactor operator to make adjustments
to the liquid fuel salt composition. This is of critical importance as the 
liquid fuel salt in a MSR will have some desired chemical state
in which the operator would like to maintain the salt. This is important for two
reasons; first to prevent salt components from precipitating out of solution
should they exceed their solubility limits, and second, the corrosion rate of 
the specialty structural nickel/iron alloys is on the order of
micro-meters per year when the reduction potential of the salt is kept near
a specific value while a slight deviation from this value can raise corrosion
rates to centimeters per year \cite{Corrosion}. 
As such, it is in the operator's best interest to keep a tight control over 
the MSR fuel salt.

Capturing all of these phenomena in a single nuclear fuel depletion code is
non-trivial and raises the question, how does the fuel salt composition in a 
molten salt fueled reactor change over time in response to both nuclear fuel
burnup and the reactor operator actions? Addressing this question is the goal
of the work presented herein. 

\subsection{Framing the problem}

The goal is to create a nuclear fuel depletion model which accounts for the
physical phenomena acting on the liquid fuel of a MSR with a flowing fuel core.
Perhaps the least quantified unknown is that of the operator's actions. What are
the operators objectives? What tools does the operator have at their disposal?
What are the measures by which the operator assess their actions? As in most
simulation problems the variable of most variance here is the human. To
approximate the human in this model they are replaced with a linear optimization
routine. If the constraints facing the operator may be approximated with linear
 relationships and if the operator's desires may be approximated as minimizing
or maximizing a given value then the choices of the operator may be predicted
via linear optimization. Ensuring that the problem constraints and optimization 
targets are all linear strikes a compromise between the ease and reproducibility
of the optimization solution and the model's adherence to the physical 
phenomena being simulated.

One goal an operator is stipulated to have is keeping specific chemical
species within the fuel salt at some relative proportion to some other
specific chemical species. This goal arises from the variable solubility of
chemical species within a given salt mixture - a solubility which changes
based both on temperature and the relative proportion of other chemical species.
Despite the complex and often polynomial nature of many chemical solubility
relationships, within a narrow window these relationships may be approximated
as linear.

A related objective an operator is stipulated to have is to maintain the fuel
salt reduction-oxidation (redox) potential at some desired value for corrosion
prevention as mentioned above. The existence of a temperature gradient within
the fuel flow loop of many MSR designs ensures that no equilibrium condition of
dissolved ions will exist. As the fuel salt, for example, strips chromium from
the structural alloys it would be possible for the reduced chromium ions to
reach such a concentration that the chemical process of corrosion would come to
a halt as chromium ions were stripped from the alloys as quickly as they
plate back on. However, as a flowing fuel salt moves through a temperature
gradient the solubility of various ions changes leading to the precipitation of
these ions onto the cold - and sometimes - hot legs of the fuel loop. Having
lost these precipitated ions the fuel salt is free to continue corroding
structural alloys elsewhere in the reactor. As such the traditional means of 
corrosion prevention - sacrificial anodes, biasing the surface with an electric
current, and controlling the activity of the corrosive species - are available
to the MSR operator. Most commonly a reducing agent, such as beryllium metal,
is added to the salt to balance the often oxidizing nature of fission. In
this manner the salt redox potential is kept at a point which will inhibit
corrosion - essentially there is very little free fluorine to pull our chromium
and a slight excess of positive ions to catch any new free fluorine which
fission may produce. Given that this redox potential is dependent on the power,
spectrum, and salt in the reactor its maintenance requires constant attention.

An important goal for all nuclear reactor
operators is to maintain the multiplication value of the reacting system. In a
traditional light water reactor this is accomplished by loading the reactor
core with far more uranium than it needs to be critical and then controlling
this excess criticality with control rods and in the case of non-boiling water
reactors, disolvable boron. In these scenarios neutrons which could have
contributed to power production are lost to control elements. In most, if not
all, MSR designs the reactors are designed to be operated with almost zero
excess reactivity, having just enough fissionable material to stay critical for
a small period of time. Either continually or in batches uranium salt may then
be added to the flowing fuel salt of the reactor under operation as these fuel
lines are near atmospheric pressure and can be accessed. In this way a MSR can
be kept critical by the operator with fewer neutrons lost to control elements.

An emerging goal, in terms of operational importance, is reactor safeguards or
the measures put in place to prevent illicit diversion of nuclear materials.
The traditional approaches to safeguards, namely inspecting and registering
fuel assemblies before they go in the core and after as well as through their
storage life, don't even apply to MSRs - there are no fuel assemblies and fuel
is added in small amounts continuously straight to the fuel line meaning that
fissile material is continually accessed, used, and moved. This makes
accountancy of such material incredibly difficult. Current efforts are examining
if there are any operational characteristics of MSRs which would provide an
early indication to plant operators such as a change in the delayed neutron
fraction in the core or perhaps a change in the chemical redox potential among
other effects. As such, in simulations the ability to model and assess the
impacts of a small diversion of material is necessary.

These considerations here form the general operational constraints of a MSR. A
method for simulating the fuel cycle of a MSR must account for all these
considerations as well as for the physical phenomena which act on MSRs
uniquely in reference to their solid-fuel counterparts.

\subsection{Previous Efforts}\label{ssec:efforts}

Various approaches have been made to create a methodology for simulating MSR
fuel cycles. Given the span of decades which separate the two major epochs of
MSR development there exist two distinct groups of methodologies. One, devised
in the 50s and 60s uses numerous approximation and simplifications and was
designed in a day when 1MB of memory was what a supercomputer could give you.
While these methodologies may inspire the works of today the actual products of
such are lost to time and the confinement of vacuum tube computers to museums.
The second was developed beginning in the late 90s with more recent efforts in
the 2010s to further expand the computational base employed by these
methodologies. A handful of authors of late have proposed approaches for
simulating MSR fuel cycles. The differences between these approaches tend to
fall into 3 categories; treatment of the multiplication factor control,
assumptions regarding fuel chemistry, and the set of possible reactor
operations.

In one of the most influential MSR fuel cycle papers to be published this
century, Aufiero unveils a custom modification to the SERPENT 2 code designed to
assist in the modelling of the European MSFR project \cite{Aufiero}. As all
developers must Aufiero made certain assumptions in the development of his
modification. For instance, he
 approximates nuclear criticality as only being dependent 
on two isotopes, chosen by the user which can be fed and removed proportionally
to the deviation from criticality. To handle salt species considerations Aufiero
ignores them, simply removing lithium for each fission product produced and
adding it for each fission product removed as seen in equation \ref{eq:auf} 
where $\phi$ is the scalar neutron flux, and $b$ is the branching ratio
from isotope $j$ into Li \cite{Aufiero}. 

\begin{equation} \label{eq:auf}
\frac{\partial N_{Li}}{\partial t} = \sum_{j} N_{j} \phi \sigma_{j \rightarrow
    Li} - \sum_{j} N_{li} \phi \sigma_{Li \rightarrow j} + \sum_{j} N_{j}
    \lambda_{j} b_{j \rightarrow Li} - \sum_{k = HM} N_{j} \phi \sigma_{kf}
    \left ( \sum_{l=FP}FY_{k \rightarrow f} \right )
\end{equation}

As will be seen in chapter \ref{ch:results} this assumption does not always hold
given the corrosion concerns of operating an MSR - concerns which Aufiero ignores
entirely. Aufiero's modification allows for the user to re-compile SERPENT to
make modifications to the nuclear decay constant of any isotope - in effect 
creating a proportional feed or removal stream. While Aufiero's work was the
first in the modern millennium to demonstrate the reactivity following aspects
of an MSR, his modifications to SERPENT 2 proved clunky and inadequate for
further pursuit.

In a scripted package Ridley implements an MSR burnup method using SERPENT 2
wrapped with Python. While Aufiero makes the hard assumption that Lithium is
removed for fission products Ridley goes the other direction and makes the
hard assumption that lithium is added for fission products - the authors
disagreeing on the oxidative or reducing potential of fission. Other than
injecting lithium Ridley largely ignores chemistry and corrosion concerns much
like Aufiero. Again, like Aufireo, Ridley instead focuses on finding the fuel
feed rate to keep the simulated reactor critical. In his method, at every
burnup step, Ridley runs dozens of Monte Carlo core simulations to estimate
the impacts of different feed rates. The proposed package then uses Python to
fit a polynomial curve to the core reactivity produced by each of the dozens
of feed rates. From this curve Ridley's method then selects the ``best" feed
rate and implements this as a batch addition to the fuel before going through
another SEPRENT Monte Carlo and burnup cycle \cite{Ridley}. 

Betzler takes a different approach overall by adding wrappers and utilities
to the reactor physics package, SCALE \cite{Betlzer}. 
Chemistry control of the simulated
system is handled by one of these wrappers. Once TRITON has finished simulating
the system of interest from a neutronics perspective the material compositions
are passed to this wrapper which adjusts the concentration of specific isotopes
to fixed user imposed limits. The resulting feeds and removals which would be
needed to achieve these values are then approximated as a proportional removal
constant, exactly like Aufiero's work, and then fed to ORIGEN. This wrapper
system only has the ability to model proportional flows who's constants do not
change over a burnup step - this necessitates rather small, 3 day, burnup steps
in order that the proportional constant used does not stray to far from the
desired result. With regards to criticality control Betzler employ's the crudest
method of all those seen, iteratively changing the concentration of a single
user specified isotope and re-running the system simulation to see if the
desired criticality condition was met. No automated system for addressing
chemistry control concerns was included with Betzler's approach. 

The approach presented herein consists of a more detailed, customizable, and
nuanced solution to the question of MSR burnup. This approach, dubbed ADER for
the \textbf{A}dvanced \textbf{D}epletion \textbf{E}xtension for 
\textbf{R}eprocessing, is a source code modification to the popular reactor
physics code SERPENT 2 which brings to the user the ability to more accurately
model and simulate MSR physics. ADER allows the user, in an abstracted and
simplified way, to model chemicals and their interdependent relationships, the
impact of material flows on system criticality, the driving factors of corrosion
, and the influences of human decisions on a nuclear system. All of this
directly integrated into SERPENT 2 with full user support including
documentation and a full test suite. Behind all of these systems the user may
see is a linear optimization kernel informing modifications to the SERPENT 2
burnup kernel.  In the following chapters ADER is introduced. First the theory 
on which ADER is based is
presented in chapter \ref{ch:meth} along with its integration into the reactor
physics monte-carlo code SERPENT 2 \cite{Jaakko}. In chapter
\ref{ch:simul} the capabilities of ADER are investigated in relation to a
hypothetical MSR fuel cycle. In chapter \ref{ch:con} the effects of ADER on
MSR fuel cycle modelling are presented as well as recommendations for next
steps. 
