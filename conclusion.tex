\chapter{Conclusions and Future Work}
\label{ch:conc}

To conclude this dissertation, we present a summary of the results and analysis delivered in the
previous chapter followed by an outlook for future work based on the preliminary endeavors
undertaken here. The ultimate research goal achieved in this work as well as the complementary 
objectives met along the way serve to contribute to the body of knowledge regarding hybrid methods
for neutral particle radiation transport in shielding and deep penetration applications.

\section{Summary and Conclusions}

In this work, the LDO equations were implemented in the Exnihilo parallel neutral particle 
radiation transport framework for the purpose of using the equations' solutions in Monte Carlo
variance reduction parameter generation via the ADVANTG software to improve the results of
simulations run with MCNP5. A small variety of test case scenarios were examined in the CADIS and
\fwc\ contexts with biasing parameters generated from flux solutions of different quadrature 
types to ascertain the LDO solutions' performance relative to unbiased Monte Carlo calculations 
as well as those with biasing parameters from standard quadrature types.

Deterministically-calculated forward and adjoint scalar flux solutions from the LDO equations 
were found to be comparable to those of standard quadrature types. The LDO equations saw their
best-performing Monte Carlo biasing parameters in the \fwc\ context. For the DLVN experimental
benchmark, LDO variance reduction parameters generated the highest Figures of Merit for two of the
six detector locations in the assembly. Of those studied here, the only other quadrature type to
achieve this was the QR quadrature set. In the case of the simplified portal monitor scenario
studied in the \fwc\ context, the LDO biasing parameters attained the highest FOM values for three
out of four detector locations. Considering results from the test case scenarios in which neutrons
were transported using the CADIS and \fwc\ methods, we suggest a coarse angular mesh for Monte 
Carlo variance reduction parameter generation based on flux solutions resultant from solving the 
LDO equations. For photon transport problems, a more refined LDO angular mesh is recommended for
generating Monte Carlo biasing parameters and achieving detector responses with high Figures of
Merit.

In general, the LDO formulation is most useful in the specific context of Monte Carlo variance
parameter generation using the \fwc\ method for photon transport problems. It is also effective in
the \fwc\ method for neutron transport problems, though somewhat less so. However, the LDO
representation is currently limited in applicability by its current implementation in the Exnihilo
framework and the ADVANTG software. The problem space available to explore is limited to those 
with vacuum boundary conditions and isotropic fixed particle sources with non-zero volume.
Adopting the LDO formulation in another radiation transport and Monte Carlo variance reduction
parameter generation framework would be of interest if the framework is flexible in allowing for
asymmetric quadrature sets to be used and if the framework allows for relative ease in 
implementing the unique features of the LDO representation such as interpolation in angle.

To conclude, we note that the novel work towards this dissertation includes the implementation of
the LDO equations in a radiation transport framework as well as the study of the resultant scalar
flux solutions' efficacy in Monte Carlo variance parameter generation in both the CADIS and \fwc\
methods. The results and analysis presented here are of interest to the community at large in that
the LDO representation treats particle scattering differently than the traditional discrete
ordinates formulation and incorporates angular information into scalar flux solutions
in a new way. This improves the performance of hybrid methods in shielding problems with highly 
anisotropic particle movement and particle streaming pathways when using the \fwc\ method, 
especially for photon transport problems.

\section{Future Work}

Various avenues of future work have become apparent over the course of this work. Some 
facets of investigation are more rudimentary and concern details regarding the implementation of
the LDO equations in a radiation transport software framework, while others are more
research-oriented questions.

We will first discuss suggested future work that concerns implementation details and studies that
may follow. As reflective boundary conditions are commonly used in both deterministic and Monte
Carlo transport methods to reduce problem space and compute time, one of the first next steps to
take would be to enable the use of reflective boundary conditions with the LDO equations in
Denovo. This is less straightforward than for standard quadrature types because of the LDO
equations' asymmetry in angle, but it should be possible using the interpolation property of the 
LDO representation. In a similar vein, enabling the use of analytic approximations of uncollided 
flux sources in combination with solving the LDO equations in Denovo would be a next logical 
development. This would enable the direct use of point sources when solving the LDO equations
through the Exnihilo framework and help to mitigate the ray effects from the point sources in the
resultant flux solutions.

Modifying the ADVANTG software to support more variety in deterministic calculations and Monte
Carlo variance parameter generation would open doors for more interesting studies with the LDO
equations as well as standard quadrature types. Specifically, implementing the use of anisotropic
particle sources would allow for studies involving commonly-used directional particle sources such
as beams. Additionally, generalizing the \fwco\ methods implemented in the ADVANTG
software such that quadrature sets that are not rotationally symmetric could be directly used in 
concert with the methods would bring about an additional channel for capturing the LDO equations'
unique scattering treatment in angle-informed scalar flux solutions used to generate Monte Carlo 
biasing parameters. Currently, the LDO equations' deterministic flux solutions could be used in 
combination with the \fwco\ methods with the use of the interpolation property of
the LDO equations. This interpolation functionality does not exist in either the Exnihilo
framework or the ADVANTG software at the time of this writing.

Broader research questions involving solving the LDO equations are of interest as well. The test
case scenarios in this work were limited to relatively small scales with respect to computational
hardware usage; studies with finer LDO angular meshes across larger hardware configurations would
be instructive to see at what subspace degree, if any, the LDO equations' flux solutions mitigate 
ray effects in relevant real-world scenarios. Additionally, we suggest testing rotated versions 
of the LDO quadrature sets to study the relationship between ordinate placement and problem 
geometry in detector response and FOM production. This would be a fairly straightforward next 
step; the point sets generated by Womersley used in this work are rotationally invariant and the 
Exnihilo framework imports the LDO quadrature sets from plain text files. So, rotation matrices or
formulae could be applied to the already-existing LDO point sets with the new rotated quadratures 
directly input to Exnihilo (and ADVANTG) for study in deterministic calculations as well as in 
Monte Carlo variance reduction parameter generation. As an alternative to conducting studies with
increased angular resolution, we suggest undertaking investigations that use the interpolatory
nature of the LDO representation once this ability has been realized in the various software
frameworks involved.
