% (This file is included by thesis.tex; you do not latex it by itself.)

\begin{abstract}

Molten salt reactors (MSRs) are a class of nuclear reactor which uses a molten 
ionic liquid as either the coolant or also as the fuel. While a 8 MWth MSR was
succesfully operated in the 1960s it was not until the early 2000s that MSRs
gained widespread attention. Since then MSRs have enjoyed plentiful research
support with such international projects as ALISIA and MOSART as well as
dedicated conferences such as the Oak Ridge National Laboratory's MSR workshop
which attracts more than 400 researchers every year. Through this considerable
research effort significant progress has been made in such areas as materials
selection, corrosion, salt purificatoins and properties idientification, as well
as general reactor design. Desipe these advances and more to date only one
general MSR fuel cycle analysis tool is available for use by the research
community and even this this tool lacks an ability to dynamically adapt itself
to a changing simulation environment as such possibly providing ansewrs of a
lower quality. In this work a method is proposed and implemented within the 
SERPENT 2 reactor physics monte-carlo code. This method, named ADER - the
Advanced Depletion Extension for Reprocessing - is a seamlessly incorporated
source code modification to the SERPENT 2 base code which allows the user
to define arbitrary collections of elements, isotopes, and chemicals.
Furthermore ADER allows the user to specify a variety of linear relationships
between these groups as well as limitations on and methods for their
movement throughout the simulated system. Lastly but certainly not least,
ADER provides to the user an optimization target. Through these structures
much of the complex chemistry, corrosion modelling, and nuclear concerns of
operating a MSR can be linearlized and solved against an optimization target,
say the decreased consumption of fuel, and passed through a linear optimization
solver, in this work the CLP library as part of the COIN-OR package, from which
an optimized system material composition and material flows solution may be 
found. ADER then uses this solution to create a brand new depletion matrix 
which SERPENT 2 then solves using the CRAM approximation method. From this
algorithm a more accurate modelling of MSR fuel cycles and physics may be
arrived at through the consideration of chemistry driven limitations,
corrosion driven limitations, nuclear driven limitations, and operator driven
limitations. Results from this implemented method indicate that ADER drives
the MSR fuel cycle simulations towards a more physically representative
result. Unfortuntely, as detailed later in this work, an underlying and
pernicious numerical instability issue was uncovered within the linear
optimization library selected for this work. Any future work on this method
must begin with the adoption of a quadruple-precision floating-point linear
optimization library over the current implementation of CLP as used in ADER.
In the following chapters an introduction to MSRs and their fuel cycle modelling
is given. Following this the theory behind ADER and its implementation within 
SERPENT2 is discussed after which the results from one of a few numerically
stable simulations is presented after which concluding remarks are given.
\end{abstract}
