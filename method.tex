\chapter{Method Description and Implementation}
\label{ch:method}

In the most direct sense ADER seeks to accomplish two tasks: 
determining an optimal material composition given a set of constraints, 
and integrating the necessary composition adjustments into a nuclear 
material evolution model. The
constraints are provided by a user and are built in a generic and
system-agnostic manner. Constraints come in three types, material abundance
constraints, nuclear constraints, and corrosion constraints. Material abundance
constraints are built using the group structures and involve limits on the
absolute and relative abundance of arbitrary collections of mass, groups, within
the material. Corrosion constraints are built around a loose approximation
of the Nernst equation as detailed in section \ref{ssec:oxi}. Nuclear
constraints are composed of minimum and maximum bounds for the system
neutron multiplication factor, $k_{eff}$, as such ensuring that mass flows
into and out of the system do not drive the system away from the desired
criticality state.

\section{Governing Equations}
Material abundance constraints are built around the concept of a \textit{group}.
A group is a list of elements and their relative proportions,
these elements themselves may or may not have specified isotopic proportions. 
A group, being a set of ratios between sets, is arbitrary in size.

From the framework provided by the concept of a group a system of 
linear equations and relationships can be built which, when taken together, 
provide an approximation of the constraints in the system to be simulated. Such
constraints could include limits on the fractional abundance of a group within
a material, bounds on the  relative abundance between groups within a material, 
the sources, sinks, and compositions of mass transfers within
the system, just to point out a few. 
Additionally, applying linear constraints to various weighted sums allows for
restrictions on approximations of material qualities such as the redox 
potential and the neutron multiplication factor. 

Ensuring that the problem constraints and optimization targets are all linear 
strikes a compromise between the ease and reproducibility and speed of the 
optimization  solution and the model's adherence to the physical phenomenon
being simulated. 
As such care was taken to preserve the linear nature of the material 
optimization problem. In the following sections the details pertaining to each
type of constraint are presented.

\subsection{Composition constraints} \label{ssec:group_eq}
A material, in this framework, is a collection of isotopes. In a nuclear 
system, it might be desirable to control a material's  composition by imposing 
specific constraints such as chemical solubility and isotopic enrichment limits.
These limits are described through the group structure. A single value for 
the fractional abundance of a group in
a material, for example the amount of trifluoride compounds in a fluoride 
salt, is represented by Equations
\ref{eq:group_def_ele} and \ref{eq:group_def_iso}

\begin{equation}
\label{eq:group_def_ele}
g_{k} = \sum \limits_{j}^{J} E_{j}^{f,k}
\end{equation} 

\begin{equation}
\label{eq:group_def_iso}
g_{k} = \sum \limits_{k}^{K} I_{k}^{f,k}
\end{equation}

where $g_{k}$ is the fractional abundance of the 
group in the host material, $E_{j}^{f,k}$ is the fractional abundance of element
$j$ in group $k$ - this elemental mass having come from the material to which
group $k$ is associated with.
$I_{k}^{f,k}$ is the fractional 
abundance of isotope $k$ in group $k$ - this isotopic mass having come from
the material to which group $k$ is assigned.
If the constraint is not a single value but a range then 
Equations \ref{eq:group_def_ele} and \ref{eq:group_def_iso} become the 
inequalities seen
in Equations \ref{eq:group_def_ele_lim} and \ref{eq:group_def_iso_lim} where
$b_{m}$ and $b_{M}$ are the lower and upper bounds, respectively.

\begin{equation}
\label{eq:group_def_ele_lim}
b_m \leq \sum \limits_{j}^{J} E_{j}^{f,k} \leq b_{M}
\end{equation} 

\begin{equation}
\label{eq:group_def_iso_lim}
b_{m} \leq \sum \limits_{k}^{K} I_{k}^{f,k} \leq b_{M} 
\end{equation}

The second principle constraint involves the relative abundance limits between 
pairs of groups; a constraint which can be used to approximate chemical 
solubility limits. Relative group abundance limits can be expressed as seen in 
Equation \ref{eq:rto_ratio} where $r_{m/M}$ indicates a relative abundance 
minimum and maximum bound, respectively.

\begin{equation}
\label{eq:rto_ratio}
r_{m} \leq \frac{g_{1}}{g_{2}} \leq r_{M} 
\end{equation}

Equation \ref{eq:rto_ratio} can be expressed linearly as two inequalities as
shown in Equations \ref{eq:rto_min} and \ref{eq:rto_max}.

\begin{equation}
\label{eq:rto_min}
-\infty \leq -g_{1} + r_{m}g_{2} \leq 0
\end{equation}

\begin{equation}
\label{eq:rto_max}
0 \leq -g_{1} + r_{M}g_{2} \leq \infty
\end{equation}

% ******************************************************************************************************

\subsection{Material flow constraints} \label{ssec:stream_eq}
Another set of constraints regards the sources, sinks, and compositions of mass
flows to, from, and between materials. These constraints are applied through 
the \textit{stream} structure; the definition of which can be
seen in Equations \ref{eq:stream_eq_ele} and \ref{eq:stream_eq_iso}.

\begin{equation}
\label{eq:stream_eq_ele}
s_{l} = \sum \limits_{j}^{J} E_{j}^{d}
\end{equation}

\begin{equation}
\label{eq:stream_eq_iso}
s_{l} = \sum \limits_{k}^{K} I_{k}^{d}
\end{equation}

Taking $s_{l}$ to be stream $l$, $E_{j}^{d}$ to be the change in the 
abundance of element $j$ in the affected material, and $I_{k}^{d}$ to be 
the change in the abundance of isotope $k$ in the affected
material.
% ******************************************************************************************************

\subsection{Oxidation constraints} \label{ssec:oxid_eq}
A weighted sum over all elements in a material with target values forms the 
next constraint which can be applied to a material (Equation \ref{eq:oxi}). 
Although the framework for this summation was implemented as a rough 
approximation to redox potential
monitoring in liquid systems (discussed more in Section \ref{ssec:oxi}) there is
no reason the weights involved could not represent some other quantity of 
interest. 

\begin{equation}
\label{eq:oxi}
O_{m} \leq \sum \limits_{j}^{J} w_{E_{j}} E_{j} \leq O_{M}
\end{equation}
%
where  $w_{E_{j}}$ is a weighting
factor which can be applied to any element.

% ******************************************************************************************************

\subsection{Reactivity constraints} \label{ssec:reactivity}
A weighted sum over isotopes in a material forms the reactivity constraint that
may be applied. This constraint is derived from the expression for the 
multiplication factor as found in Equation \ref{eq:reac}:

\begin{equation}
\label{eq:reac}
k_{eff} = P_{NL} \frac{\sum\limits^{M}_{m}\phi_m\omega_m\nu\Sigma_{f}^{m}}
{\sum\limits^{M}_{m}\phi_m\omega_m   \Sigma_{a}^{m}}
\end{equation}
%
where $\phi_m$, $\omega_m$, $ \nu\Sigma_{f}^{m}$ and $ \Sigma_{a}^{m}$ are, 
respectively, the scalar neutron flux, the volume fraction, the spectrum 
averaged neutron production cross section, and the macroscopic absorption cross
section for each material $m$. $P_{NL}$ is the neutron non-leakage probability.
In ADER, the ability to control $k_{eff}$ is limited to the case of a single 
neutron multiplying material; take $\nu\Sigma_{f}=0$ for every material but the
multiplying material $M$ and as such Equation \ref{eq:reac} can be rewritten as
follows:

\begin{equation}
\label{eq:reac_one_material}
k_{eff} = P_{NL} \frac{\phi_M\omega_{M}\Sigma_{a}^{M}}{\sum\limits^{M}_{m}\phi_m\omega_m \Sigma_{a}^{m}} \frac{\nu\Sigma_{f}^{M}}{\Sigma_{a}^{M}}
\end{equation}

The probability of a neutron being absorbed in the multiplying material is 
defined as follows:

\begin{equation}
\label{eq:abs_prob}
P_{A} = P_{NL} \frac{\phi_M\omega_{M}\Sigma_{a}^{M}} {\sum\limits^{M-1}_{m}\phi_{m}\omega_m\Sigma_{a}^{m}}
\end{equation}

Then $k_{eff}$ can be calculated as:

\begin{equation}
\label{eq:reac_modified}
k_{eff} = P_{A} \frac{\nu\Sigma_{f}^{M}}{\Sigma_{a}^{M}} = P_{A} \frac{\sum\limits^{I}_{i}\nu\Sigma_{f}^{i}}{\sum\limits^{I}_{i}\Sigma_{a}^{i}}
\end{equation}

where $\nu\Sigma_{f}^i$, and $\Sigma_{a}^i$ are, respectively, the spectrum 
averaged neutron production cross section, and the absorption cross section for
every isotope $i$ in the multiplying material $M$. This relation is expected to
hold for simulations in which there is a dominant reactive material and for 
which $\nu\Sigma_{f} \approx 0$ for all other materials.

In this case, given lower and upper bounds for the multiplication factor of the
system, $k_{eff}^{min}$ and $k_{eff}^{max}$ respectively, 
Equation \ref{eq:reac_modified} can be  made linear as in 
Equations \ref{eq:k_min} and \ref{eq:k_max}.

\begin{equation}
\label{eq:k_min}
0 \geq \frac{k_{eff}^{min}}{P_{A}} \sum \limits_{k}^{K} \sigma_{a}^{k} I_{k} - \sum \limits_{k}^{K} \nu^{k} \sigma_{f}^{k} I_{k}
\end{equation}

\begin{equation}
\label{eq:k_max}
0 \leq \frac{k_{eff}^{max}}{P_{A}} \sum \limits_{k}^{K} \sigma_{a}^{k} I_{k} - \sum \limits_{k}^{K} \nu^{k} \sigma_{f}^{k} I_{k}
\end{equation}

A key assumption of this linearization process is that 
$\frac{\partial P_{A}(m...M)}{\partial M} = 0$ when in truth
$P_{A}$ is a function of the composition of material $M$. 
The impacts of this approximation are expected to be quite small but it will 
affect all simulations, more so those with strong leakage effects.
%
% ******************************************************************************************************

\subsection{The optimization method} \label{ssec:opt}
Equations \ref{eq:group_def_ele_lim}, \ref{eq:group_def_iso_lim}, 
\ref{eq:rto_min}, \ref{eq:rto_max}, \ref{eq:stream_eq_ele}, 
\ref{eq:stream_eq_iso}, \ref{eq:oxi}, \ref{eq:k_min}, and \ref{eq:k_max} 
demonstrate the linearity of the problem.  Given a linear
set of equations, and a set which in various configurations could be
under-constrained, over-constrained, or equal, a unique solution can not be
guaranteed at all times. As such the best `solution' is an optimization route
through which the addition of an optimization target can guarantee a unique
solution. In respect to the the ease of implementation, the ease of use, and
the comprehensibility of the simulation results,
linear optimization or linear programming as it is sometimes known, was
chose as the optimization method. The linear
programming problem is represented by a sparse matrix which is manipulated to 
produce the optimal solution given an optimization target 
(details in Section \ref{ssec:opt_matrix}).


\section{Implementation in SERPENT 2} \label{sec:implementation}
Utilizing the linear relationships described in 
Section \ref{sec:equations}, ADER brings to SERPENT 2 the ability for
users to define desired relationships between constituent units of a material 
and to define material flows within
the system. Additionally ADER provides tools for constraints based upon the 
elemental composition of a material
as well as constraints relating to the multiplication factor of a system. 
In this Section the realization of these tools in SERPENT 2 is detailed.


% ******************************************************************************************************

\subsection{Groups} \label{ssec:groups}
A \textit{group} is an elementary structure in the ADER framework as described 
in Section \ref{ssec:group_eq}.
Elaborating further, a group is composed of a fixed set of elements, with or 
without specified isotopics, with fixed abundances relative to the group as a 
whole, e.g.,  a group could be made to specify that it is one part uranium and 
four parts chlorine (uranium tetrachloride). Furthermore, the uranium could be 
specified to be 4.95\% \ce{^{235}U} and 95.05\% \ce{^{238}U}. 
A group is not a \textit{material} in the SERPENT meaning of it, but rather a 
material constituent that is connected to a material by set relations. 
For example, the user can define the material FLiBe as 2LiF-Be\ce{F_2}, 
then define two groups as \ce{LiF} and \ce{BeF_2} and then specify the 
constraint that the material FLiBe maintains a  2:1 ratio between the two 
groups regardless of any other occurring change. As stated in Section 
\ref{sec:implementation} a group can be applied to any control volume and 
itself has no inherent density associated with it.

ADER provides several means to define the relationships between groups, and 
between  materials and groups. To define relationships between groups (in a
material) a range of relative abundances between any two groups
may be defined for an arbitrary number of group pairs. To define relationships
between the material and the related groups, a range of absolute abundances of 
a group in a material may be specified for an arbitrary number of groups. 
The governing equations of these relationships are found in 
Section \ref{ssec:group_eq}. To facilitate modelling
of chemical compounds with related and possibly interchangeable forms, a group
in ADER may also be formed from the linear combination of any other groups
previously defined. These three simple mechanisms can be combined to model
various chemical situations and form a core component of the conditions for
optimality placed on a material. 

Say, for example, that a material is desired to have three to four times as
much eutectic FLiBe salt to uranium fluoride salts, both 
uranium trifluoride and tetrafluoride. Additionally, it is desired that uranium
tetrafluoride be more than 100 times as abundant as uranium trifluoride. Setting
these as constraints for the material can be accomplished with four groups and
two relative abundance constraints. The following groups are needed: a uranium 
trifluoride group, a uranium tetrafluoride group, a uranium fluoride group 
obtained as a summation of the previous groups, and a FLiBe group. 
A relative abundance constraint is placed between
the FLiBe group and the uranium fluoride group, and a relative
constraint is placed between the two uranium fluoride salt groups. With those
six constructs a solubility constraint on a family of related compounds is put
into effect. This is just one example of many restraints and conditions which
can be modeled with the ADER group structures and the relationships between
them.

Finally, related to the group structure in ADER, is the concept of 
\textit{free} versus \textit{controlled} elements or isotopes. 
In ADER, for a given material, whether or not an element or isotope should be 
completely accounted for by the groups
which possess these constituents or be allowed to have free portions
not locked up in the group structure is something that can be specified. For 
example, if a material were to have a uranium tetrafluoride group the 
fluorine in that material would be \textit{controlled} when all the fluorine in
that material were required to be accompanied by 0.25 uranium atoms. If the
fluorine content of the material were left to be \textit{free} then the 
material would be allowed to have a fluorine to uranium ratio less than 
0.25---indicating that not all fluorine in the material is bound in a U\ce{F4} 
compound.

% ****************************************************************************

\subsection{Streams} \label{ssec:streams}
A \textit{stream} is both the workhorse and the end-goal of ADER.
There are two classes of streams in ADER: group-class and table-class. 
Group-class streams are options. They represent pathways available to ADER to 
move mass into, out of,
and between SERPENT materials with the goal of bringing their compositions to an
optimal state. Table-class streams are prescriptive. They are directions to
ADER to move specific types and amounts of mass from and to specific materials
---the results of table-class streams are factored into the material 
composition before determination of optimality allowing the effects of 
group-class streams to reflect the consequences of table-class stream effects. 
All streams have a set of common attributes
provided by the user: a source, a sink, and the behavior in time of the stream.
For the majority of streams a source and a sink are optional, but at least one 
of the two must be provided. Missing sources are treated as infinite supplies 
of whatever substance is needed; missing sinks are treated much like 
sinks---endless consumers of disposed mass. In terms of their behavior in time,
ADER supports three types of streams: discrete type stream transfers happen
between burnup steps as step changes; continuous type stream transfers occur as
a steady rate of mass transfer over the length of a burnup step; proportional
type streams modify the decay constant
of isotopes, even to the point of making the decay constant a production
constant if that is what is called for. A notable feature related to streams
is the option to require that inflows match outflows for specified materials;
this dramatically simplifies the set of constraints needed to model a practical
system in which the mass is not simply allowed to vary between 0 and $\infty$.

Group-class streams have an additional attribute the user is required to set: 
the ADER group which defines the substance the stream will move. These streams
are given no set amount of mass transfer. Rather, through its optimization 
process
ADER determines the amount of mass transfer each group-class stream should
have. Table-class streams have two additional attributes the user is required to
set: the ADER transfer table to be used and a positive value, denoted
$c^{s}$.
Transfer tables in ADER are user defined lists of selected elements
and isotopes all of which have some value attached to them, 
denoted $c_{k}^{t}$. Multiplying the
value from the transfer table with the value given in the table-class stream
definition gives the fraction of the whole for an individual isotope or
element that will be moved by the table-class stream per unit time over
the next burnup step. For proportional type streams the value
produced by this multiplication will be added to the decay constant of the
appropriate isotopes, or subtracted given the stream's relation to the 
material in question.The value of splitting
the table-class stream mass transfer rates into two numbers lies with MSR
modelling. In many proposed MSR designs there is some fuel treatment procedure
which is applied to some fraction of the fuel salt, represented by the value
given in the table-class stream definition. This treatment procedure removes
specific elements with differing effectiveness, as represented by the value
given to each element and isotope contained in a transfer table.

Streams, group-class or table-class, have clear applicability to MSR modelling.
ADER's optimization routines, in this instance, should be thought of as
the reactor
operator with the streams representing those mass flows in and out of a 
reactor that
the operator may plan; such as an addition of lithium fluoride for maintaining
a desired salt condition or the addition of \ce{^{233}U} for criticality
control. Table-class streams provide a means not only to model possible
fuel salt reprocessing options but also natural process which change the
composition of fuel salts such as the escape of noble gas fission products. 
Outside of MSR modelling streams find other applications ranging from geological
repository modelling to biological radiation dose analysis.

% ******************************************************************************************************

\subsection{Oxidation control} \label{ssec:oxi}
As mentioned in Section \ref{ssec:oxid_eq} the oxidation control portion of 
ADER is a weighted sum over
the elements in a material with bounds for the evaluation of the sum set by 
the user. In the oxidation table structure a complete list
of elements with their expected average oxidation state, or whichever weighting
factor is used, in the desired material is given. This structure was designed 
with MSR operations in mind. The redox potential of a flowing liquid
is a key parameter of interest in many liquid fuel reactor designs. It is far 
beyond the scope of ADER, and even SERPENT 2, to be determining the redox 
potential of a chemical mixture through simulation. 
That said, it is suspected that this feature will greatly ease the
burden on most MSR simulations as most proposed fuel salts have a dominant
anion which bonds with near everything else to the exclusion of other
compounds. Combining this tool with the principles from the Nernst equation,
Equation \ref{eq:nernst}, bounds for the average redox potential of a molten
salt can be set---a key metric in controlling corrosion in molten salt systems.
In Equation \ref{eq:nernst} $\epsilon_{i}$ is the redox potential,
$\epsilon_{i}^{o}$ is the redox potential in the standard state, $R$ is the gas
constant, $T$ is the temperature, $z$ is the number of electrons received by the
oxidizing agent, $j$ is Faraday's constant, $[oxid]$ is the activity of the 
oxidized species while $[red]$ is the activity of the reduced species.
Conversions between activity and concentration are required but approximations 
may be sufficient; this task is left to the user.

\begin{equation}
\label{eq:nernst}
    \epsilon_{i} = \epsilon_{i}^{o} + \frac{RT}{z \j}\ln\left(\frac{[oxid]}{[red]}\right)
\end{equation}

% ******************************************************************************************************

\subsection{Reactivity control} \label{ssec:reactivity}
The multiplication factor of a nuclear system is regularly of interest
and often desired, in a reactor, to be of a specific value.
As such users may set system wide $k$-eigenvalue
constraints through ADER. When such constraints are applied ADER
incorporates Equations \ref{eq:k_min} and \ref{eq:k_max} into
the linear optimization problem so that the effects of mass transfers
within the system can be constrained by their predicted impacts on
the multiplication factor of the system. The relevant cross section information,
for all isotopes in an ADER material is retrieved from SERPENT 2 and used to 
fill in Equations \ref{eq:k_min} and \ref{eq:k_max}. 

Even if Equation \ref{eq:reac} is exact, $k_{eff}$ is only ever approximated;
not just from the error inherent in Monte Carlo simulations, but from
ignoring that every term in Equation \ref{eq:reac} is non-linearly dependent
on composition. As such it is expected that the reactivity control feature of
ADER will only behave well for small changes in composition that do not have a
large effect on the neutron flux. Finally, more of a limitation than an
assumption, ADER has no means to measure the effect on reactivity resulting
from changes in one material interacting neutronincally with another. As such 
ADER's reactivity control feature only works in situations where one 
material is the dominant driver of criticality in a system. Partially 
to address these shortcomings, ADER offers the user the option of setting the
number of max iterations allowed per burnup step to determine the reactivity 
effects of
ADER's streams. At the end of each such iteration a Monte Carlo cross section
calculation is repeated to assess the effects of ADER's actions.

% *****************************************************************************************************


\subsection{The optimization target} \label{ssec:opt_target}
Having covered the many constraints available to be placed
on the model, the missing piece to the linear optimization 
problem is an optimization target. To this end ADER allows the user
to set the optimization direction, minimization or maximization, as well as
the optimization target from which the user can select such options, as a
specific group in a specific material, a specific stream, 
all material transfers, specific material transfers, and others detailed in the
ADER user manual \ref{app:um}.


\subsection{The optimization matrix} \label{ssec:opt_matrix}

The CLP library expects a linear programming matrix from ADER---one built from
all the constituent equations in the ADER scheme. Before the equations 
presented earlier in this section can be incorporated into the matrix a note
about the effects of table-class streams should be made. As mentioned in Section
\ref{ssec:streams} table-class streams are prescriptions, meaning the mass 
flows they
stipulate are going to happen. This information makes it into the linear
programming matrix in the form of adjustments to the bounds of specific rows.
These adjustments are denoted
as $r_{x}$ where $r$ is the net positive increase in the abundance of
component $x$ as caused by all table-class streams. It should be noted that
adjustments calculated for proportional removal table-class streams are only
approximations, and sometimes poor approximations, of the actual amount of an
isotope or element that will be removed as nuclear processes change the 
abundance of isotopes in a way that ADER is currently unaware of.

Figure \ref{fig:opt_matrix}
depicts the scheme for constructing the linear programming matrix. Column bounds
, seen above the label describing what the column represents, are given as
are row bounds which are seen to the left of the label describing which equation
the row represents. For the sake of brevity
the matrix in Figure \ref{fig:opt_matrix} is for one material only though many
materials may be involved in such a matrix should they be linked together
by shared mass transfers. In which case the only variables shared between materials are the
group-class streams and the stream equations they are a part of are the only
coupling equations; aside from transfers by table-class streams but those
are only represented in the linear programming matrix, they are handled by
other routines all together. If a second material were to be included in this
matrix then, perhaps, the stream entries in the third and fourth columns would
have non-zero coefficients for some $E_{j}^{d}$ and $I_{k}^{d}$ rows of the 
second material.

Working down the matrix row by row the first row encountered represents 
Equation \ref{eq:rto_min} with arbitrary groups $g_{1}$ and $g_{2}$ whereas 
the next row 
down represents Equation \ref{eq:rto_max}. The third row, what will be referred
to as an elemental future row, represents the atom balance for element $j$ where
$f^{n}_{E_{j}^{f}}$ is the fractional proportion of element $j$ in group $n$.
The novel column involved here is an elemental future column whose inclusion
in the same row closes the equation.  
The bounds for this row are those for a
free element or those elements which are permitted to have portions
of the element not tied up in declared group structures. In the case of a
controlled element, those whose complete abundance must be accounted for by
group structures, the lower bound is changed to zero. 
The fourth row, an elemental
delta row, represents the change in the abundance of element $j$
as caused by all group-class streams where
$f^{n}_{E_{j}^{f}}$ is the fractional proportion of element $j$ in stream $n$. 
Of course the elemental delta column 
is involved to close the balance. The fifth row, or balance row, is what ties
together $E_{j}^{f}$ and $E_{j}^{d}$. The bounds, $\alpha$ and $\beta$,
are equal and represent $E_{j}^{c} + r_{E_{j}^{f}}$ constituting an atom
balance ``in time'' where $E_{j}^{c}$ represents the present fractional
abundance of element $j$. The fifth row requires, straightforwardly, that the
future amount of an element be equal to the current amount plus any delta,
or change, in the element's abundance. The sixth row is an isotopic balance row
requiring that the abundance of an element be equal to the abundance of its
constituent isotopes. 
The following three rows, the seventh, eighth, and ninth,
are the isotopic versions of the elemental future, delta, and balance rows 
where $f$ values
are for the isotopic fractional proportions.
$\gamma$ and $\delta$ are
equal and represent $I_{k}^{c} + r_{I_{k}^{c}}$ where $I_{k}^{c}$ represents
the present fractional abundance of isotope $k$. In the tenth and eleventh rows
Equations \ref{eq:k_max} and \ref{eq:k_min} find representation with $\eta$
and $\theta$ respectively representing terms of the expanded sum found in the
referenced equations; $\frac{k_{eff}^{min}}{P_{A}} \sigma_{a}^{k} - \nu^{k}
\sigma_{f}^{k}$ and
$\frac{k_{eff}^{max}}{P_{A}} \sigma_{a}^{k} - \nu^{k}
\sigma_{f}^{k}$. 
The twelfth row represents Equation \ref{eq:oxi}, accounting for the
contributions of the future quantity of elements to a material's averaged
oxidation state.
The thirteenth row, or Pres row, exists
when the user instructs ADER to balance inflows with outflows. The
pres row requires that the net stream transfers in a material come to zero. The
effects of table-class streams are captured in $\upsilon$ and $\omega$ as seen
in Equation \ref{eq:upsilon_bound} where $s^{t}$ is a table class stream 
abundance value. The final row is the optimization, or Opt row. This row
indicates to the simplex routine which variables to minimize or maximize. In
Figure \ref{fig:opt_matrix} the opt row is indicating that $g_{1}$ is the
optimization target. The direction of optimization, maximization or
minimization, is a parameter which the user passes.
    
    \begin{equation*}
    \label{fig:opt_matrix}
        \centering
        \begin{blockarray}{cccccccccc}
                               &                   & [b_{m},b_{M}]     &
            [0, \infty)        & [0,\infty)        & [0,\infty)        &
            [0, \infty)        & (-\infty,\infty)  & [0,\infty)        &
            (-\infty,\infty)  \\ 
                               &                   & g_{1}             &
            g_{2}              & s_{1}             & s_{2}             &
            E_{j}^{f}          & E_{j}^{d}         & I_{k}^{f}         &
            I_{k}^{d} \\
                               &                   &                   &
                               &                   &                   &
                               &                   &                   &
             \\ 
            \begin{block}{cc[cccccccc]}
            {(-\infty,0]}      & \text{Eq.}\ref{eq:rto_min} & -1       &
            r_{m}              &                   &                   &
                               &                   &                   &
             \\
            {[0,\infty)}       & \text{Eq.}\ref{eq:rto_max} & -1       &
            r_{M}              &                   &                   &
                               &                   &                   &
             \\
            {(-\infty,0]}      & E_{j}^{f}         & f_{E_{j}^{f}}^{1} &
            f_{E_{j}^{f}}^{2}  &                   &                   &
            -1                 &                   &                   &
             \\
            {[0,0]}            & E_{j}^{d}         &                   &
                               & f_{E_{j}^{f}}^{1} & f_{E_{j}^{f}}^{2} &
                               & -1                &                   &
             \\
            {[\alpha, \beta]} 
                               & E_{j}^{b}         &                   &
                               &                   &                   &
            1                  & -1                &                   &
             \\
            {[0,0]}            & E_{j}^{i}         &                   &
                               &                   &                   &
            -1                 &                   & 1                 &
             \\
            {(-\infty,0]}      & I_{k}^{f}         & f_{I_{k}^{f}}^{1} &
            f_{I_{k}^{f}}^{2}  &                   &                   &
                               &                   & -1                &
             \\
            {[0,0]}            & I_{k}^{d}         &                   &
                               & f_{I_{k}^{f}}^{1} & f_{I_{k}^{f}}^{2} &
                               &                   &                   &
            -1 \\
            {[\gamma, \delta]}
                               & I_{k}^{b}         &                   &
                               &                   &                   &
                               &                   & 1                 &
            -1 \\ 
            {[0, \infty)}      & \text{Eq.}\ref{eq:k_max}&             &
                               &                   &                   &
                               &                   & \eta             &
             \\
            {(-\infty, 0]}     & \text{Eq.}\ref{eq:k_min} &            &
                               &                   &                   &
                               &                   & \theta            & 
             \\
            {[O_{m},O_{M}]}    & \text{Eq.}\ref{eq:oxi} &              &
                               &                   &                   &
             o_{E_{j}^{f}}     &                   &                   &
             \\
            {[\upsilon,\omega]} & \text{Pres}      &                   &
                               & 1                 & 1                 &
                               &                   &                   &
             \\
                               & \text{Opt}        & 1                 &
                               &                   &                   &
                               &                   &                   &
             \\
            \end{block}
        \end{blockarray}
    \end{equation*}
    Figure 2.0: A depiction of the Simplex optimization matrix.

\begin{equation}
\label{eq:upsilon_bound}
\upsilon = \omega = -\sum \limits_{s'}^{S'} s_{s'}^{t}
\end{equation}



% ******************************************************************************************************
% ******************************************************************************************************
% ******************************************************************************************************

\subsection{Solution and limitations of the linear optimization problem} \label{ssec:sol}
To solve the linear programming problem ADER employs the
CLP library from the COIN-OR project \cite{lougee-heimer_common_2003}. 
CLP is a double-precision linear optimization solver utilizing the Simplex 
algorithm. Sandia National Laboratory noted in their report on open-source
linear solvers that CLP was by far the fastest, most accurate, and most
capable of the solvers they tested and that it performed on the same
order of magnitude for any metric when compared against commercial solvers
\cite{gearhart_comparison_2013}.
Once ADER has constructed the
sparse matrix representing the linear optimization problem
and packed this matrix into a dense column major format said
matrix is handed off to the CLP simplex solution routines. 
CLP solves the linear programming problem and returns back a
vector containing the value of the objective function as well as the values
all the variables take in the optimal solution. The key pieces of information 
from this process are the values of the stream abundances. However, these
stream abundance values are burdened by two key limitations both related to the
time-independence of the linear optimization problem. 

The first limitation can be entirely avoided if all streams in a given
simulation have identical behavior in time. If streams with differing behavior
in time affect a common material the optimization solution may not be true at
any or all points in time for that solution interval. In a simple example,
imagine that material A requires four parts fluorine from stream B and one part
uranium from stream C. Take stream B to be a discrete type stream and stream C
to be a continuous type stream. Neglecting any nuclear depletion the optimal
composition will not be realized until the end of the burnup step when stream C
has delivered all of its uranium. Additionally, a previously unexpected amount
of fluoride without any corresponding uranium will appear in the material 
suddenly when the effects of stream B are applied.
Simulations involving streams with mixed time
behavior may find recourse with shorter burnup steps but mixing streams with
differing time behavior is ill-advised in general.

The second limitation arises from the effects of nuclear depletion. Should
nuclear depletion act upon a constituent of the linear optimization solution
the solution may not hold following the effects of nuclear depletion. This
limitation will be most strongly felt in simulations for which an isotope
with a large rate of change in its concentration is also a key constituent of
an optimization problem, particularly one with less flexible constraints. 
Shorter
burnup steps which minimize the integrated effect of an isotope's rate of change
may reduce the degree to which the actual composition diverges from the ideal
composition.

The greatest limitation of the solution turns out not to be related to the
physics of the problem - but to the floating point precision of computers. 
As put forward in 
\cite{STANFORD} the limitations of machine precision become critical in
SIMPLEX algorithms written with 64 bits or less of precision and with
variables having magnitudes at or lower than $10^{-6}$. This issue is easily
resolved by employing a SIMPLEX solver which uses quadruple or higher precision
for floating point numbers. 

The impact of this limitation is difficult to understate or pin down. Whether or
not a small number will cause the failure of a SIMPLEX solution depends on the
remainder of the coefficients in the optimization problem and how the SIMPLEX
algorithm goes about finding the solution. In use-testing many simulations of
postulated nuclear systems were able to be run with ADER out to thousands of
years of effective reactor operation. In many more cases the simulations fail
to find a SIMPLEX solution and exit before a year of effective reactor
operation has passed.

Any linear transformation applied to the optimization problem preserves the
overall problem of $x >> y$ where $x$ and $y$ are given values in the
optimization problem. Furthermore, in many nuclear simulations several isotopes
have significant importance to the purpose of the simulations and yet routinely
have atomic densities below $10^{-6}$ relative to their host material atomic
density - \ce{^{6}Li} being a common example.

Considering the limitations of machine precision on the efficacy of ADER, until
a SIMPLEX algorithm employing quadruple or higher precision for floating point
arithmetic is incorporated ADER must be considered stochastically functional
--- certainly a less than optimal state. While the authors of \cite{STANFORD}
provide their quad-precision code, it is written in FORTRAN which would 
necessitate the construction of a wrapper function. Additionally their code 
takes a much more detailed and error prone format than does CLP - as such 
necessitating  not insignificant structural modifications to ADER. A significant
development effort would be needed to either implement the FORTRAN library or
to update the CLP library to quad-precision.


% ******************************************************************************************************

\subsection{Material depletion} \label{ssec:burn}
Following the solution of the optimization problem discrete type streams have
their effects applied before the burnup step begins. A Monte Carlo simulation is
then run and if the multiplication factor is outside of the user defined bounds
(and iterations remain, as set by the user) another optimization solve will be
executed except the changes already made by discrete type streams remain. 
Beyond this, ADER modifies the burnup matrix inside of
SERPENT 2 to reflect the effects of streams. The coefficients in the burnup
matrix are those in the Bateman equation as seen in Equation \ref{eq:Bateman} 
for one energy group and zero dimensional case, 
where $N$ is the number density of nuclide $n$, $t$ is time, $b_{m \to n}$ is 
the branching ratio for the decay of nuclide $m$ into $n$, $\lambda$ is the 
decay constant for its sub-scripted nuclide,
$q$ goes over all neutron induced absorption reactions for a given isotope, 
$a_{m \to n}^{q}$ is the branching ratio for isotope $m$ into $n$ due to
reaction $q$, $\sigma_{x}^{y}$ is the effective microscopic
cross section of reaction $x$ for isotope $y$, $\phi$ is the
scalar neutron flux, $d$ denotes all
transmutation reactions for a given isotope, $R_{n}(t)$ is a
fractional removal (or addition) rate for isotope $n$ at
time $t$, and $F_{n}(t)$ is a feed (or removal) amount for isotope
$n$ at time $t$. 

A highly truncated burnup scheme can be
seen in Figure \ref{fig:burn_matrix} in which there are two isotopes,
\ce{^{233}U} and \ce{^{135}Xe}, and two streams; $S_{c}$ representing a
continuous stream with a constant injection rate and $S_{p}$ representing a
proportional stream with a transfer rate dependent upon the concentration
of the substances to be transferred. There are, of course, two matrices as well.
The burnup matrix to the left holding the coefficients of the Bateman equation
and the second, to the right, holding the initial concentrations of isotopes
and the values for the streams. The first column of the first row gives
the creation and destruction of \ce{^{233}U} which is dependant on the
concentration of \ce{^{233}U} with $\Gamma$ representing nuclear destruction as
seen in equation \ref{eq:Gamma_def}. The third column of the first row holds
the fraction of stream $S_{c}$ that \ce{^{233}U} comprises. These entries
together describe the evolution of \ce{^{233}U} in the given system. In the
second row $\Xi$, as seen in Equation \ref{eq:Xi_def}, represents the production
of \ce{^{135}Xe} from \ce{^{233}U}. In the second column of the second row are
the processes dependant on the concentration of \ce{^{135}Xe}. $\Upsilon$
represents the proportional rate constant as determined by the multiplication
of $c_{\ce{^{135}Xe}}^{t}$ and $c^{s}$ whereas $\Theta$ is given by Equation \ref{eq:Theta_def}. The third row is blank as the abundance of a continuous type
stream, $h_{S_{c}}$, does not change over a burn step. The fourth row is an
addition specific to ADER and not found in the Bateman equations; rather, this
line, and the lines it represents, exists to keep track of the amount of an
isotope that a proportional stream moves simply to provide this information
to the user. The system of matrices seen in Figure \ref{fig:burn_matrix} is
solved by SERPENT 2 providing updated isotopic abundances and proportional 
stream transfer amounts.

    \begin{equation}
    \label{eq:Bateman}
    \begin{split}
        \frac{\mathrm{d}N_{n}(t)}{\mathrm{d}t} = & \sum \limits_{m}^{M} 
        b_{m \rightarrow n} \lambda_{j} N_{j}(t) + \\
        & \sum \limits_{m}^{M}
        \sum \limits_{q}^{Q} a_{k \rightarrow i}^{q}
        \sigma_{q}^{k} \phi(t) N_{k}(t) - \\
        & N_{n}(t) \lambda_{i} - \sum \limits_{d}^{D}
        \sigma_{d}^{n} \phi(t) N_{n}(t) - \\
        & R_{n}(t) N_{n}(t) + F_{n}(t)
    \end{split}
    \end{equation}

    \begin{equation}
    \label{fig:burn_matrix}
        \begin{blockarray}{cccccc}
             &
            \ce{^{233}U} &
            \ce{^{135}Xe} &
            S_{c} &
            S_{p} &
            \mathbb{N} \\
             &
             &
             &
             &
             &
             \\ 
        \begin{block}{c[cccc][c]}
            \ce{^{233}U} &
            -\lambda_{\ce{^{233}U}} + \Gamma &
             &
            f^{S_{c}}_{\ce{^{233}U}} &
             &
            N_{\ce{^{233}U}} \\
            \ce{^{135}Xe} &
            \Xi &
            -\lambda_{\ce{^{135}Xe}} + \Upsilon + \Theta &
             &
             &
            N_{\ce{^{135}Xe}} \\
            S_{c} &
             &
             &
             &
             &
            h_{S_{c}} \\
            S_{p} &
             &
             \Upsilon &
             &
             &
             0\\
        \end{block}
        \end{blockarray}
    \end{equation}

\begin{equation}
\label{eq:Gamma_def}
\Gamma = - \sum \limits_{d}^{D} \sigma_{d}^{\ce{^{233}U}} \phi
\end{equation}

\begin{equation}
\label{eq:Xi_def}
\Xi = b_{\ce{^{233}U} \rightarrow \ce{^{135}Xe}} \lambda_{\ce{^{233}U}} + \sum
\limits_{q}^{Q} a_{\ce{^{233}U} \rightarrow \ce{^{135}Xe}} \sigma_{q}^{
\ce{^{233}U}} \phi
\end{equation}

\begin{equation}
\label{eq:Theta_def}
\Theta =-\sum \limits_{d}^{D} \sigma_{d}^{\ce{^{135}Xe}} \phi
\end{equation}

% ******************************************************************************************************

\subsubsection{Iterations} \label{sssec:iter}
The Bateman equation Eq.\ref{eq:Bateman} is not a standalone description of the
isotopic evolution of a nuclear system; rather, it is tightly coupled with the
scalar neutron flux. Considering that the solution of the system of matrices
in Figure \ref{fig:burn_matrix} does not solve for the scalar neutron flux
it is clear that the solution, if for nothing else, is an approximation. Many
nuclear material evolution schemes iterate between solutions of the neutron
flux and the Bateman equations within the same burnup step---SERPENT 2 is no 
different. Although the purpose of
this section is not to investigate the burnup solution routines of SERPENT 2, 
those
can be seen in \cite{leppanen_burnup_2009}, the iteration scheme employed by
these routines could affect ADER. In truth, ADER is compatible with any
iteration scheme employed by SERPENT 2 consequently in part to a limitation 
of ADER ---
due to the mix of continuous and discrete streams convergence of an iteration
scheme for optimization involving nuclear processes on the fuel is impossible
to guarantee. At the present time ADER only iterates to check the reactivity 
component of its solution, as mentioned in subsection \ref{ssec:burn}. 
After the linear programming
matrix has been built and solved, the effects of discrete type streams are
applied to the pertinent SERPENT 2 materials. Following the application 
of discrete type streams the transport sweep is re-run. If the system analog 
$k_{eff}$ is within bounds as set by the user, program-flow will continue on to
the building and solution of the burnup matrices; otherwise, ADER re-builds
and re-solves the linear programming matrix and applies the new discrete type
streams on top of the changes made by the previous iteration of discrete 
type streams. These iterations happen at the beginning of every burn step in 
which the Bateman equations will be solved. Other than these actions, ADER 
does not
interact with Serpent 2 burnup iterations schemes. A flowchart roughly outlining
SERPENT 2 and ADER interaction is seen in Figure 2.1.

\subsection{Algorithm implementation} \label{ssec:algorithm}
Concerning the software engineering aspects of ADER's creation the most 
useful resources to any curious individual are the API, user manual, and
source code provided in appendices \ref{app:api} and \ref{app:um}
respectively while the source code is online along with the system tests
described later in this paragraph. 
ADER was developed in a test-driven environment and as such
has more than 150 unit tests supporting its development as found in the file
\texttt{testcases.c}. More than a dozen integration tests can be found
within the code contained in functions which begin with the prefix
\texttt{TEST}. Lastly, more than 20 system tests can be found online
at \verb|www.github.com/ddwooten/ADER_pub/System_Testing/|. ADER's
directory structure includes no levels below the \texttt{/src} folder as
the parent project, SERPENT 2, does not either. ADER's code style adheres
closely to that of SERPENT 2 but adopts a more readable use of white space.
The input for ADER is directly integrated into the SERPENT 2 input and uses
the same style.
 
\begin{figure} \label{fig:flow_chart}
\begin{centering}
\begin{tikzpicture}[node distance = 1.5cm, every text node part/.style={font=\small}]
\node (start) [startstop] {SERPENT 2 Starts};
\node (input) [io, below=0.5cm of start] {User input};
\node (dec1) [decision, below=0.5cm of input] {Are there burnup steps remaining?};
\node (first) [process, below=0.5cm of dec1] {Monte-Carlo transport sweep};
\node (end) [startstop, right=0.5cm of dec1] {End};
\node (ader_build) [process, below=0.5cm of first] {ADER builds the optimization matrix};
\node (ader_solve) [process, below=0.5cm of ader_build] {CLP solves the optimization
matrix producing values for group-class streams};
\node (ader_apply) [process, below=0.5cm of ader_solve] {Discrete type streams apply changes
to material compositions};
\node (transport) [process, below=0.5cm of ader_apply] {Monte-Carlo transport sweep};
\node (dec2) [decision, below=0.5cm of transport] {Is iteration count at maximum OR $k_{eff}^{min} \leq k_{eff}^{analog} \leq k_{eff}^{max}$};
\node (burn) [process, below=0.5cm of dec2] {Burnup};

\draw [arrow] (start) -- (input);
\draw [arrow] (input) -- (dec1);
\draw [arrow] (dec1) -- (first);
\draw [arrow] (dec1) -- node[anchor=west] {yes} (first);
\draw [arrow] (dec1) -- (end);
\draw [arrow] (dec1) -- node[anchor=north] {no} (end);
\draw [arrow] (first) -- (ader_build);
\draw [vecArrow] (input) -- ([shift={(3cm,0cm)}]input.east) |- node[anchor=north, text width=3cm] {optimization parameters, material compositions} (ader_build.east);
\draw [vecArrow] ([shift={(-2.5cm,0cm)}]first.south) -- node[anchor=east] {cross sections}([shift={(-2.5cm,0cm)}]ader_build.north);
\draw [arrow] (ader_build) -- (ader_solve);
\draw [arrow] (ader_solve) -- (ader_apply);
\draw [vecArrow] (ader_apply) -- ([shift={(2cm,0cm)}]ader_apply.east) |- node[anchor=north] {material compositions} ([shift={(0cm,0cm)}]transport.east);
\draw [arrow] (ader_apply) -- (transport);
\draw [vecArrow] (transport) -- node[anchor=north] {cross sections} ([shift={(-2cm,0cm)}]transport.west) |- ([shift={(0cm,0cm)}]ader_build.west);
\draw [arrow] (transport) -- (dec2);
\draw [arrow] (dec2) -- node[anchor=north] {no} ([shift={(-3cm,0cm)}]dec2.west) |- ([shift={(-1cm,0.25cm)}]ader_build.west) -- ([shift={(0cm,0.25cm)}]ader_build.west);
\draw [arrow] (dec2) -- node[anchor=west] {yes} (burn);
\draw [arrow] (burn) -- ([shift={(-3.5cm,0cm)}]burn.west) |- ([shift={(-3.5cm,0cm)}]dec1.west) -- (dec1);
\end{tikzpicture}
\end{centering}
\caption{A simplified schematic of interactions between SERPENT 2 and ADER. Black lines
represent process flow while double-lined arrows highlight the flow of specific information.}
\end{figure}
