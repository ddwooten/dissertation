\chapter{Conclusions and Future Work}
\label{ch:conc}

In the last twenty years molten salt reactors have seen an incredible surge of
global attention as evidenced in such international programs as MOSART, ALISIA,
and the LF-TMSR out of China, as well as increased research attention as
evidenced by the continuing success of Oak Ridge National Laboratory's MSR
workshop which now averages over 400 attendees from across the globe yearly. 
While significant investment and progress has been made in nearly all aspects
of MSR development from material selection, salt purification and property
measurement, licensing and general reactor design, comparatively little
attention has been paid to fuel cycle analysis. 

The majority of attempts in
this area have fallen short of producing widely applicable results largely due
to the limiting effects of specific assumptions such as the salt species
involved or the coarseness of the solution. In this work a method and
implementation of a general approach for modelling and evaluating molten salt
reactor fuel cycles is proposed. This approach is based upon a linear
optimization routine designed to approximate the limitations of the chemistry
and nuclear concerns of reactor operations while optimizing the driving concerns
of the reactor operators. This approach, named ADER for the Advanced Depletion
Extension for Reprocessing, is built into the reactor physics Monte-Carlo code
SERPENT 2.  

ADER brings to the users of SERPENT 2 the ability to model several aspects of
MSR operations and fuel cycle analysis through the introduction of several vital
utilities. The first among these is the ability to define collections of
elements, isotopes and chemicals as well as the ability to set absolute and 
relative abundance constraints between any set of these collections, called
groups. In order that ADER might push material compositions towards those which
satisfy the group constraints ADER provides to the user the ability to define
mass transfers into, out of, and between materials in a SERPENT 2 simulation. 
To ensure that these mass transfers do not disturb the desired neutron
multiplication factor in the system ADER allows the user to set neutron 
multiplication maximum and minimum values. During the material optimization
phase ADER estimates the reactivity impact of its selected mass transfers and
adjusts these transfers to keep the system within the desired bounds. Furthermore
ADER provides the ability for the user to set the desired bounds of the averaged
oxidation state of materials in a simulation such that ADER will keep its
selected mass transfers from disturbing the redox potential of the material,
a critical value in determining the effects and extent of corrosive activity. 
Bringing all of this together is an easy to use and directly integrated user
interface supported by a full suite of tests as well as documentation. 

In chapter \ref{ch:results} the results of a simple simulation employing a 
\ce{LiF-BeF2-ThF4-UF4} salt mixture in an infinite and homogeneous medium are
presented. While this simulation did not cause CLP to crash and error out,
nonetheless the results obtained from the simulation present a mixed view of
ADER's implementation. ADER is certainly seen to influence the outcome of the
depletion simulation, adding uranium salts to maintain the minimum value of the
neutron multiplication factor in the system. However, occurring around burnup
step 219, ADER's corrections fail to bring the neutron multiplication value
of the system up to at least the minimum value. ADER did maintain the necessary
amount of fluorine in the system to bind to all primary salt constituents.
However, the mass flows ADER employed to accomplish this took on non-physical
values. With regards to the corrosion monitoring feature of ADER, these
limitations were completely ignored by the simulation for an unknown reason
likely having to do with the inherent numerical instability within the
optimization library. Despite these shortcomings this simulation has undoubtedly
shown the value of an algorithm for incorporating the concerns which ADER
addresses into a nuclear fuel depletion simulation. As put forth in chapter
\ref{ch:method} the implementation of a quad-precision linear optimization solver
should, according to \cite{STANFORD}, resolve the numerical instabilities
plaguing ADER - at which point the algorithm is expected to show great utility
in molten salt reactor analysis.

Any future
work going forth on this project would need to begin with the implementation
of said floating-point quadruple-precision linear optimization solver in the 
place
of the current CLP implementation. Following such a development it is expected
that ADER will be capable of simulating the wide variety of parameters which its
input and structure allow. Additional improvements could be found through the
incorporation of an iteration scheme whereupon the effects of nuclear burnup
on the isotopics of the fuel over a burnup step can be approximated as a
proportional removal stream on the whole system in the linear optimization
scheme such that the approximated total effects of nuclear burnup are considered
in material optimization. 

Overall the impact of ADER is clear in that it represents a more complete
approach to MSR fuel cycle modelling.
Given ADER's extensive test suite, documentation, and modular construction, it
is not unimaginable that the above mentioned improvements may one day be made.
In such a future the impact of ADER would certainly be greater.
Despite this shortcoming ADER has shown that a linear optimization scheme can
be effectively applied to the chemistry, nuclear, and operational concerns of a
molten salt reactor in such a way as to predict the future fuel composition and 
nuclear characteristics of the system.
